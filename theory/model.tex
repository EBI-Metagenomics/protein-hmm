\section{Model}

\begin{definition}\label{def:mp}
  A Markov process is a stochastic process $Q_1, Q_2, \dots$ for which
  \begin{equation*}
    p(Q_t=q_t \gv Q_1=q_1, Q_2=q_2, \dots, Q_{t-1}=q_{t-1}) = p(Q_t=q_t \gv Q_{t-1}=q_{t-1}).
  \end{equation*}
  The possible values of $Q_t$ form a finite set $\set{Q}$ called the state space.
\end{definition}

\begin{definition}\label{def:hmm}
  Let $\set{A}$ be a non-empty finite set of symbols. Let $Q_1, Q_2, \dots$ be a Markov process and
  let $S_1, S_2, \dots$ be a stochastic process for which
  \begin{equation*}
    p(S_t\in\set{A} \gv Q_1=q_1, Q_2=q_2, \dots, Q_t=q_t) = p(S_t\in\set{A} \gv Q_t=q_t).
  \end{equation*}
  The pair $(Q_t, S_t)$ is a hidden Markov model (HMM) with alphabet $\set{A}$.
\end{definition}

Let $\arr{z}=z_1z_2 \dots z_\ell$ be a sequence of symbols from alphabet $\set{A}$ of a given HMM.\@
The marginal likelihood of $\arr{z}$ is defined by
\begin{equation}\label{eq:hml}
  \mathrm{ML}(\arr{z}) \eqdef p(S_1=z_1, S_2=z_2, \dots, S_\ell=z_\ell).
\end{equation}

The standard HMM definition is often extended to include states that do not emit symbols. Those
states are referred to as silent states and are useful to describe a missing alignment position, for
example. This section goes a step further by defining a more general hidden Markov model that
accounts for states that instead emit sequence of symbols of variable length, including zero-length
sequences.

\begin{definition}
  Let $\set{A}$ be a non-empty finite set of symbols, $k\in\field{N}_0$, and define
  $\set{T}=\bigcup_{i=0}^k\set{A}^i$.
  Let $Q_1, Q_2, \dots$ be a Markov process and let $S_1, S_2, \dots$ be a stochastic process for
  which
  \begin{equation*}
    p(S_t\in\set{T} \gv Q_1=q_1, Q_2=q_2, \dots, Q_t=q_t)
    = p(S_t\in\set{T} \gv Q_t=q_t).
  \end{equation*}
  The pair $(Q_t, S_t)$ is an invisible Markov model (IMM) with alphabet $\set{A}$ and
  limit $k$.
\end{definition}

Let $\arr{z}=z_1z_2 \dots z_\ell$ be a sequence of symbols from alphabet $\set{A}$ of a given IMM.\@
The marginal likelihood of $\arr{z}$ cannot be written as in \Cref{eq:hml} since
we've lost the the order association between symbols and steps.
Instead, the marginal likelihood is given by
\begin{equation}\label{eq:ml}
  \mathrm{ML}(\arr{z}) \eqdef \sum_{t=1}^{\infty} p(S_{1..t}=\arr{z}, S_{t+1}\neq \emptyset),
\end{equation}
where $S_{1..t}$ denotes the concatenation of the random variables $S_1$, $S_2$, $\dots$, and $S_t$.
The inequality $S_{t+1}\neq \emptyset$ is necessary to avoid evaluating the same probability twice.
The infinity summation is also needed because of the possibility of existing cycles in the IMM
formed by states that are able to emmit empty sequences.

\begin{remark}
  The notation $p(S_{1..t}=\arr{z})$ is equal to the summation of the probabilities of every valid
  association between the random variables $S_1, S_2, \dots, S_t$ and the subsequences of $\arr{z}$,
  including empty subsequences. For example, let $a$ be a sequence composed of a single symbol. We
  have $p(S_{1..2}=a) = p(S_1=a, S_2=\emptyset) + p(S_1=\emptyset, S_2=a)$ by definition, where
  $\emptyset$ denotes an empty sequence.
\end{remark}

\subsection{Viterbi}

Let us consider first the Viterbi method applied to HMMs for a given sequence $\arr{z}$ of length
$\ell$.
Let
\begin{equation*}
  \viterbi(q_t) \eqdef \umax{q_{1..t-1}} \{ p(S_{1..t}=z_{1..t}, Q_{1..t}=q_{1..t}) \}
\end{equation*}
be the so-called Viterbi score. The $\max$ operator in the previous definition is the maximum across
all possible associations $Q_1=q_1, Q_2=q_2, \dots, Q_{t-1}=q_{t-1}$. Viterbi score is therefore the
maximum probability among all state paths that ends in $q_t$ and generates the prefix $z_{1..t}$
from sequence $\arr{z}$. The function domain of $\viterbi(q_t)$ is such that $t \in \{1, 2, \dots,
\ell\}$ and $q_t \in \set{Q}$.

Viterbi score can also be defined in a recursive fashion as follows:
\begin{equation*}
\begin{split}
  \viterbi(q_t)
  &= p(S_t=z_t \gv Q_t=q_t) \umax{q_{1..t-1}}
    \{ p(Q_t=q_t \gv Q_{t-1}=q_{t-1}) p(S_{1..t-1}=z_{1..t-1}, Q_{1..t-1}=q_{1..t-1}) \} \\
  &= p(S_t=z_t \gv Q_t=q_t) \umax{q_{t-1}}
    \{ p(Q_t=q_t \gv Q_{t-1}=q_{t-1})
    \umax{q_{1..t-2}} \{ p(S_{1..t}=z_{1..t-1}, Q_{1..t}=q_{1..t-1}) \} \} \\
  &= p(S_t=z_t \gv Q_t=q_t) \umax{q_{t-1}} \{ p(Q_t=q_t \gv Q_{t-1}=q_{t-1})
    \viterbi(q_{t-1}) \},
\end{split}
\end{equation*}
for $t>1$; and
\begin{equation*}
  \viterbi(q_1) = p(S_1=z_1 \gv Q_1=q_1) p(Q_1=q_1).
\end{equation*}

The simplicity of the notation $\viterbi(q_t)$ is no longer enough to define the Viterbi score for
IMMs as the index $t$ does not unambigously defines a prefix of $\arr{z}$. We will also make use of
an additional parameter $F_t\in \{0, 1, \dots, k\}$: $F_t=f_t$ means that $f_t$ is the length of the
sequence emitted by $S_t$.
For the sake of notation clarity, let us define $z_{i(f_t)..i} \eqdef
z_{i-f_t+1..i}$ as the $f_t$-length tail of a sequence $z_{1..i}$. The Viterbi score of an IMM
regarding a sequence $\arr{z}$ of length $\ell$ is defined by
\begin{equation}\label{eq:viterbi}
  \viterbi_{i}(q_t,f_t) \eqdef
    \umax{\substack{q_{1..t-1}\\f_{1..t-1}}}
    \{
      p(S_{1..t}=z_{1..i}, Q_{1..t}=q_{1..t} ~;~ F_{1..t}=f_{1..t})
    \}.
\end{equation}

The $\max$ operator in \Cref{eq:viterbi} is the maximum across all possible associations
$Q_1=q_1, Q_2=q_2, \dots, Q_{t-1}=q_{t-1}$ and $F_1=f_1, F_2=f_2, \dots, F_{t-1}=f_{t-1}$. Viterbi
score is therefore the maximum probability among all state paths that ends in $q_t$ emitting
$z_{i(f_t)..i}$ and generates the prefix $z_{1..i}$ from sequence $\arr{z}$. The function domain of
$\viterbi_{i}(q_t,f_t)$ is such that $t \in \{1, 2, \dots\}$, $i \in \{0, 1, \dots, \ell\}$, $f_t
\in \{0, 1, \dots, \min\{i, k\}\}$, and $q_t \in \set{Q}$.

Viterbi score for IMM can also be defined in a recursive way as follows:
\begin{equation*}
  \begin{split}
    \viterbi_{i}(q_t,f_t)
    &= p(S_t=z_{i(f_t)..i} \gv q_t)
      \uumax{q_{t-1}}{f_{t-1}}
      \{
        p(q_t \gv q_{t-1})
        \uumax{q_{1..t-2}}{f_{1..t-2}}
        \{
          p(S_{1..t-1}=z_{1..i-f_t}, q_{1..t-1} ~;~ f_{1..t-1})
        \}
      \}\\
    &= p(S_t=z_{i(f_t)..i} \gv q_t)
      \uumax{q_{t-1}}{f_{t-1}}
      \{
        p(q_t \gv q_{t-1}) \viterbi_{i-f_t}(q_{t-1},f_{t-1})
      \},
  \end{split}
\end{equation*}
for $t>1$; and
\begin{equation*}
  \viterbi_{f_1}(q_1,f_1) = p(S_1=z_{1..i} \gv Q_1=q_1) p(Q_1=q_1).
\end{equation*}
