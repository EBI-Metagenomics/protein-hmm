\section{Model}

\begin{definition}\label{def:mp}
  A Markov process is a stochastic process $Q_1, Q_2, \dots$ for which
  \begin{equation*}
    p(Q_t=q_t \gv Q_1=q_1, Q_2=q_2, \dots, Q_{t-1}=q_{t-1}) = p(Q_t=q_t \gv Q_{t-1}=q_{t-1}).
  \end{equation*}
  The possible values of $Q_t$ form a finite set $\set{Q}$ called the state space.
\end{definition}

\begin{definition}\label{def:hmm}
  Let $\set{A}$ be a non-empty finite set of symbols. Let $Q_1, Q_2, \dots$ be a Markov process and
  let $S_1, S_2, \dots$ be a stochastic process for which
  \begin{equation*}
    p(S_t\in\set{A} \gv Q_1=q_1, Q_2=q_2, \dots, Q_t=q_t) = p(S_t\in\set{A} \gv Q_t=q_t).
  \end{equation*}
  The pair $(Q_t, S_t)$ is a hidden Markov model (HMM) with alphabet $\set{A}$.
\end{definition}

Let $\arr{z}=z_1z_2 \dots z_\ell$ be a sequence of symbols from alphabet $\set{A}$ of a given HMM.\@
The marginal likelihood of $\arr{z}$ is defined by
\begin{equation}\label{eq:hml}
  \mathrm{ML}(\arr{z}) \eqdef p(S_1=z_1, S_2=z_2, \dots, S_\ell=z_\ell).
\end{equation}

The standard HMM definition is often extended to include states that do not emit symbols. Those
states are referred to as silent states and are useful to describe a missing alignment position, for
example. This section goes a step further by defining a more general hidden Markov model that
accounts for states that instead emit sequence of symbols of variable length, including zero-length
sequences.

\begin{definition}
  Let $\set{A}$ be a non-empty finite set of symbols, $k\in\field{N}_0$, and define
  $\set{B}=\bigcup_{i=0}^k\set{A}^i$.
  Let $Q_1, Q_2, \dots$ be a Markov process and let $S_1, S_2, \dots$ be a stochastic process for
  which
  \begin{equation*}
    p(S_t\in\set{B} \gv Q_1=q_1, Q_2=q_2, \dots, Q_t=q_t)
    = p(S_t\in\set{B} \gv Q_t=q_t).
  \end{equation*}
  The pair $(Q_t, S_t)$ is an invisible Markov model (IMM) with alphabet $\set{A}$ and
  limit $k$.
\end{definition}

Let $\arr{z}=z_1z_2 \dots z_\ell$ be a sequence of symbols from alphabet $\set{A}$ of a given IMM.\@
The marginal likelihood of $\arr{z}$ cannot be written as in Eq.~\eqref{eq:hml} since
we've lost the the order association between symbols and steps.
Instead, the marginal likelihood is given by
\begin{equation}\label{eq:ml}
  \mathrm{ML}(\arr{z}) \eqdef \sum_{t=1}^{\infty} p(S_{1..t}=\arr{z}, S_{t+1}\neq \emptyset),
\end{equation}
where $S_{1..t}$ denotes the concatenation of the random variables $S_1$, $S_2$, $\dots$, and $S_t$.
The inequality $S_{t+1}\neq \emptyset$ is necessary to avoid evaluating the same probability twice.
The infinity summation is also needed because of the possibility of existing cycles in the IMM
formed by states that are able to emmit empty sequences.

\begin{remark}
  The notation $p(S_{1..t}=\arr{z})$ is equal to the summation of the probabilities of every valid
  association between the random variables $S_1, S_2, \dots, S_t$ and the subsequences of $\arr{z}$,
  including empty subsequences. For example, let $a$ be a sequence composed of a single symbol. We
  have $p(S_{1..2}=a) = p(S_1=a, S_2=\emptyset) + p(S_1=\emptyset, S_2=a)$ by definition, where
  $\emptyset$ denotes an empty sequence.
\end{remark}

\subsection{Viterbi}

Let us consider first the Viterbi method applied to HMMs for a given sequence $\arr{z}$ of length
$\ell$.
Let
\begin{equation*}
  \viterbi(q_t) \eqdef
  \begin{dcases*}
    \umax{q_{1..t-1}} \{ p(S_{1..t}=z_{1..t}, Q_{1..t}=q_{1..t}) \} & if $t>1$ \\
    p(S_1=z_1 \gv Q_1=q_1) p(Q_1=q_1)                               & if $t=1$
  \end{dcases*}
\end{equation*}
be the so-called Viterbi score. It is the maximum probability among all state paths that ends in
$q_t$ and generates the prefix $z_{1..t}$ from sequence $\arr{z}$.
The function domain of $\viterbi(q_t)$ is such that $t \in \{1, 2, \dots, \ell\}$.

Viterbi score can also be defined in a recursive fashion as follows:
\begin{equation*}
\begin{split}
  \viterbi(q_t)
  &= p(S_t=z_t \gv Q_t=q_t) \umax{q_{1..t-1}}
    \{ p(Q_t=q_t \gv Q_{t-1}=q_{t-1}) p(S_{1..t-1}=z_{1..t-1}, Q_{1..t-1}=q_{1..t-1}) \} \\
  &= p(S_t=z_t \gv Q_t=q_t) \umax{q_{t-1}}
    \{ p(Q_t=q_t \gv Q_{t-1}=q_{t-1})
    \umax{q_{1..t-2}} \{ p(S_{1..t}=z_{1..t-1}, Q_{1..t}=q_{1..t-1}) \} \} \\
  &= p(S_t=z_t \gv Q_t=q_t) \umax{q_{t-1}} \{ p(Q_t=q_t \gv Q_{t-1}=q_{t-1})
    \viterbi(q_{t-1}) \},
\end{split}
\end{equation*}
for $t>1$.

The notation $\viterbi(q_t)$ under the HMM context conveys three crucial information: (i) we are at
step $t$ of the HMM process; (ii) we are querying something about some state $q_t \in \set{Q}$
regarding (iii) the prefix $z_{1..t}$. The same notation under the IMM context loses the information
(iii): the index $t$ no longer unambigously defines a prefix. We will instead use
$\viterbi_{q_t,f_t}(i)$ to define the Viterbi score, for which $i$ defines the prefix $z_{1..i}$. We
also make use of index $f_t$, which determines the sequence length that state $q_t \in \set{Q}$ is
being queried about.
In summary, the notation $\viterbi_{q_t,f_t}(i)$ under the IMM context will convey four crucial
information:
\begin{itemize}
  \item We are at step $t$ of the IMM process;
  \item We query something about some state $q_t \in \set{Q}$;
  \item The query regards the prefix $z_{1..i}$;
  \item And we conjecture that the last $f_t$ symbols of $z_{1..i}$ have been emitted at step $t$ by
    state $q_t$.
\end{itemize}
For the sake of notation clarity, let us denote $z_{i(f_t)..i} \eqdef z_{i-f_t+1..i}$ as the
$f_t$-length tail of a sequence $z_{1..i}$.

The Viterbi score of an IMM regarding a sequence $\arr{z}$ of length $\ell$ is defined by
\begin{equation*}
  \viterbi_{i}(q_t,f_t) \eqdef
  \begin{dcases*}
    \umax{\substack{q_{1..t-1}\\f_{1..t-1}}}
    \{
      p(S_{1..t}=z_{1..i}, Q_{1..t}=q_{1..t} ~;~ F_{1..t}=f_{1..t})
    \}                                                                  & if $t>1$\\
    p(S_1=z_{1..f_1} \gv Q_1=q_t) p(Q_1=q_1)                            & if $t=1$
  \end{dcases*}
\end{equation*}
It is the maximum probability among all state paths that ends in
$q_t$ emitting $z_{i(f_t)..i}$ and generates the prefix $z_{1..i}$ from sequence $\arr{z}$.
The function domain of $\viterbi_{i}(q_t,f_t)$ is such that $t \in \{1, 2, \dots\}$, $i \in \{0, 1,
\dots, \ell\}$, $f_t \in \{0, 1, \dots, i\}$, and $q_t \in \set{Q}$.

Viterbi score for IMM can also be defined in a recursive way as follows:
\begin{equation*}
  \begin{split}
    \viterbi_{i}(q_t,f_t)
    &= p(S_t=z_{i(f_t)..i} \gv q_t)
      \uumax{q_{t-1}}{f_{t-1}}
      \{
        p(q_t \gv q_{t-1})
        \uumax{q_{1..t-2}}{f_{1..t-2}}
        \{
          p(S_{1..t-1}=z_{1..i-f_t}, q_{1..t-1} ~;~ f_{1..t-1})
        \}
      \}\\
    &= p(S_t=z_{i(f_t)..i} \gv q_t)
      \uumax{q_{t-1}}{f_{t-1}}
      \{
        p(q_t \gv q_{t-1}) \viterbi_{i-f_t}(q_{t-1},f_{t-1})
      \},
  \end{split}
\end{equation*}
for $t>1$.

\subsection{Discussion}

For computational reasons, it would be useful to have an upper bound on the summation of the
marginal likelihood.
We will define a type of IMM that has such a feature.

\begin{definition}
  A cycle is any probable sequence of states that starts and ends with the same state.
\end{definition}

\begin{definition}
  A quiet state is a state that has a non-zero probability of emitting an empty sequence.
\end{definition}

\begin{definition}
  A quiet cycle is any cycle having only quiet states.
\end{definition}

\begin{corollary}
  Let $M$ be the number of states of the IMM.\@ If it has no quiet cycles, any sequence of $M$ states
  will have emitted at least one symbol.
\end{corollary}

If IMM has no quiet cycles, there is a finite limit for the marginal likelihood summation.
